
\documentclass[a4paper,12pt]{article}
\usepackage[margin=1in]{geometry}

\usepackage[T2A]{fontenc}			% кодировка
\usepackage[utf8]{inputenc}			% кодировка исходного текста
\usepackage[english,russian]{babel}	% локализация и переносы
\usepackage{graphicx}                % Математика
\usepackage{amsmath,amsfonts,amssymb,amsthm,mathtools} 
\usepackage{mathtext}
\usepackage[T2A]{fontenc}
\usepackage[utf8]{inputenc}

\usepackage{wasysym}

%Заговолок
\author{Бичина Марина 
группа Б04-005 1 курса ФЭФМ}
\title{Лабораторная работа №1.3.3 \\ Измерение вязкости воздуха по течению в тонких трубках}
\date{\today}


\begin{document} % начало документа

\maketitle
\newpage

\section{Аннотация}

\paragraph{Цель работы:} 
\begin{enumerate}
\itemsep0em
\item 
экспериментально исследовать свойства течения газов по тонким трубкам при различных числах Рейнольдса
\item 
 выявить область применимости закона Пуазейля и с его помощью определить коэффициент вязкости воздуха 
\end{enumerate}
\paragraph{Оборудование:}
\begin{enumerate}
\itemsep0em
\item 
система подачи воздуха (компрессор, проводящие трубки)
\item
газовый счетчик барабанного типа
\item
спиртовой микроманометр с регулируемым наклоном
\item
 набор трубок различного диаметра с выходами для подсоединения микроманометра
\item
  секундомер
\end{enumerate}
\section{Теоретическая часть:}
\paragraph{}
Силы вязкого (<<внутреннего>>) трения возникают между соседними слоями жидкости при ее движении, а также со стороны стенок трубы. Описываются они с помощью закона Ньютона:
\begin{equation}
\tau_{xy} = -\eta \frac{dv_{x}}{dy}
\end{equation}
где:
\begin{enumerate}
\itemsep0em
\item $\tau_{xy}$ -- касательное напряжение
\item $v_{x}$ -- скорость течения вдоль оси х
\item y -- координата y  
\item $\eta$ -- коэффициент динамической вязкости (вязкость) жидкости
\end{enumerate}
\subsection{Описание установки:}
\paragraph{}

\subsection{Контрольные вопросы:}
\begin{enumerate}
\itemsep0em
\item 
\end{enumerate}
\section{Ход работы:}
\begin{enumerate}
\renewcommand{\labelenumii}{\arabic{enumii})}
\itemsep0em
\item
\end{enumerate}
\section{Выводы:}
\begin{enumerate}
\item
\end{enumerate}
\end{document}